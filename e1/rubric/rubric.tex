\documentclass[11pt,a4paper]{article}

\usepackage{tadoc}

\usepackage{amsmath,amssymb,amsfonts}
%\usepackage{xcolor}
%\usepackage{graphicx}


\usepackage{minted}
\usemintedstyle{autumn}
\setminted{linenos,breaklines,tabsize=4,xleftmargin=1.5em}

\usepackage{dirtree}

%\renewcommand{\multirowsetup}{\centering} 

\usepackage{pdfpages}


\title{VE572 --- Methods and Tools for Big Data}
\subtitle{Midterm Rubric}
\author{\href{mailto:liuyh615@sjtu.edu.cn}{Yihao} and \href{mailto:AuroraZYJ@sjtu.edu.cn}{Yanjun}}
\semester{Summer}
\year{2019}
\blockinfo{
	\begin{center}
		\textbf{This document should be kept secret.}
	\end{center}	
}

% whether or not to display the instructor line
\noinstructor

%\pagenumbering{gobble}

\begin{document}

\maketitle

\section*{Exercise 2 --- MapReduce}

\subsection*{1. Determine all the FOFs in the following toy example.}

\inputminted{shell}{../data.txt}

\subsection*{2. Write the Hadoop pseudocode
for the first MapReduce Job. Assume a simple input text file
with a list of names on each line, the user as first field followed by all his friends. For the
output we expect a simple text file where each line is composed of a user and a FOF followed
by the number of friends they have in common.}

\inputminted{java}{../src/main/java/com/ve572/e1/FindFOF.java}

\subsection*{3. Write the Hadoop pseudocode
for the second MapReduce job. Assume the previous output
file as input, and as output a simple text file where each line is composed of a user and all his
FOF ordered with respect to the number of common friends; for each FOF also display the
number of common friends.}


\inputminted{java}{../src/main/java/com/ve572/e1/CountFOF.java}


\end{document}

