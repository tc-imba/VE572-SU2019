\documentclass[11pt,a4paper]{article}

\usepackage{tadoc}

\usepackage{amsmath,amssymb,amsfonts}
%\usepackage{xcolor}
%\usepackage{graphicx}


\usepackage{minted}
\usemintedstyle{autumn}
\setminted{linenos,breaklines,tabsize=4,xleftmargin=1.5em}

\usepackage{dirtree}

%\renewcommand{\multirowsetup}{\centering} 

\usepackage{pdfpages}


\title{VE572 --- Methods and Tools for Big Data}
\subtitle{Midterm Rubric}
\author{\href{mailto:liuyh615@sjtu.edu.cn}{Yihao} and \href{mailto:AuroraZYJ@sjtu.edu.cn}{Yanjun}}
\semester{Summer}
\year{2019}
\blockinfo{
	\begin{center}
		\textbf{This document should be kept secret.}
	\end{center}	
}

% whether or not to display the instructor line
\noinstructor

%\pagenumbering{gobble}

\begin{document}

\maketitle

\section*{Exercise 1 --- Big data or not big data? [20 marks]}

This part will count at most 20 marks in total.

\subsection*{1. What is the data size? [2.5+5=7.5 marks]}

\begin{enumerate}
	\item 1G-10G: Spark without Hadoop
	\item 10G-100G: Mapreduce with Hadoop
	\item 100G+: Spark with Hadoop
\end{enumerate}

\subsection*{2. What is the data type? [2.5+5=7.5 marks]}

\begin{enumerate}
	\item Real time: Spark
	\item Batch: Mapreduce
\end{enumerate}

\subsection*{3. What does the company want to do with the data? [2.5+5=7.5 marks]}

\begin{enumerate}
	\item Search: Drill
	\item Implement algorithms: Spark
\end{enumerate}

\subsection*{4. Other reasonable answers [5 marks]}

\section*{Exercise 2 --- MapReduce [40 marks]}

\subsection*{1. Determine all the FOFs in the following toy example. [(2/3)*15=10 marks]}

\inputminted{shell}{../fof.txt}

\subsection*{2. Write the Hadoop pseudocode for the first MapReduce Job. Assume a simple input text file with a list of names on each line, the user as first field followed by all his friends. For the output we expect a simple text file where each line is composed of a user and a FOF followed by the number of friends they have in common. [15 marks]}

\begin{itemize}
\item The overall MapRuduce Structure [5 marks]
\item The first MapReduce functionality [5 marks]
\item The second MapReduce functionality [5 marks]
\end{itemize}

\inputminted{java}{../src/main/java/com/ve572/e1/FindFOF.java}

\subsection*{3. Write the Hadoop pseudocode for the second MapReduce job. Assume the previous output file as input, and as output a simple text file where each line is composed of a user and all his FOF ordered with respect to the number of common friends; for each FOF also display the number of common friends. [15 marks]}

\begin{itemize}
\item The overall MapRuduce Structure [7.5 marks]
\item The MapReduce functionality [7.5 marks]
\end{itemize}

\inputminted{java}{../src/main/java/com/ve572/e1/CountFOF.java}

\section*{Exercise 3 --- Course questions [35 marks]}
\subsection*{1. HDFS [15 marks]}
\subsubsection*{a) What is the default replication level in HDFS? [1 mark]}
The default replication factor is 3.
\subsubsection*{b) Parallel is often seen as more efficient than serial. When writing a file in HDFS, blocks are first sent to a DataNode which forwards them to another, which sends them to another, and so on…Why is this process not parallelised, i.e. send blocks to all the DataNodes at the same time? [4 marks]}

\begin{itemize}
\item If the blocks are sent to all the DataNodes at the same time, it will be slow due to the limited throughput. [2 marks]
\item Thus when writing a file in HDFS, the blocks are sent to a DataNode at a time. Once the blocks are written in a DataNode, it is no need to forward them to other DataNodes. [2 marks]
\end{itemize}

\subsubsection*{c) Explain how to find a file in HDFS. [3 marks]}
\begin{enumerate}
	\item Each datanode announces the blocks it has. [1 mark]
	\item The namenode keeps all the information in its memory. [1 mark]
	\item When a write occurs an entry is added to the edit log. [1 mark]
\end{enumerate}

\subsubsection*{d) What are the namespace image and edit log? [3 marks]}

\begin{itemize}
\item Namespace image: Stroing the entire file system namespace,  including the mapping of blocks to files and file system properties. [1.5 marks]
\item Edit log: Recording every change that occurs to file system metadata, such as creating a new file in HDFS and changing the replication factor of a file. [1.5 marks]
\end{itemize}



\subsubsection*{e) Is it possible to have several NameNodes in a cluster? If so explain how it works. [4 marks]}

\begin{itemize}
\item It is possible. [1 mark]
\item We can use HDFS federation. [1 mark] 
\item Split the filesystem over several independent namenodes. Each namenode has a namespace and its own pool of blocks [1 mark]
\item A namespace with a block pool is called namespace volume and a datanode is not attached to a specific namespace volume. [1 mark]
\end{itemize}

\subsection*{2. YARN [8 marks]}
\subsubsection*{a) Explain how an application is launched and run using YARN. [3 marks]}
\begin{enumerate}
	\item Application client submit a YARN application to Resource Manager.
	\item Resource Manager contacts Node Manager to launch a new container and run Application Master in it. [1 mark]
	\item Application Master asks Resource Manager for allocating the resources. [1 mark]
	\item Application manager gets the resources information from Resource Manager and it launches the container through other Node Manager. [1 mark]
\end{enumerate}

\subsubsection*{b) Would you recommend the fair or capacity scheduler? Explain the when and why. [5 marks]}
\begin{itemize}
\item
\begin{itemize}
\item Fair scheduling is a method of assigning resources to applications such that all apps get, on average, an equal share of resources over time. [1 mark]
\item It is a good default for small to medium sized clusters [0.5 mark]
\item since it is more flexible and allows for jobs to consume unused resources in the cluster. [1 mark]
\end{itemize}
\item
\begin{itemize}
\item Capacity scheduler is designed to run Hadoop applications as a shared, multi-tenant cluster in an operator-friendly manner while maximizing the throughput and the utilization of the cluster. [1 mark]
\item It's generally used on large clusters with lots of different workloads with different needs [0.5 mark]
\item since it can give each organization capacity guarantees.[1 mark]
\end{itemize}
\end{itemize}

\subsection*{3. Briefly explain the similarities and differences between MapReduce and Spark. [7 marks]}

\begin{itemize}
\item Similarity: Both are highly scalable and can be used in a cluster. [1 mark]
\item MapReduce: 
\begin{itemize}
\item MapReduce takes two stages to process data, Map and Reduce. It reads and writes from disk and thus slower and has high latency. [1 mark]
\item It uses replication for fault tolerance, which significantly increase the completion times for operations with a single failure. [2 marks]
\end{itemize}
\item Spark: 
\begin{itemize}
\item
Spark is lightning fast cluster computing tool. It is much faster than MapReduce and has low-latency computing. [1 mark] 
\item
It uses RDDs and DAG for fault tolerance. If an RDD is lost, it is easy to recompute a new one by using the original transformations. [2 marks]
\end{itemize}
\end{itemize}

\subsection*{4. Drill [5 marks]}

\subsubsection*{a) What is Zookeeper, and why is it a core component of Drill’s strategy? [2 marks]}

\begin{itemize}
\item ZooKeeper is a centralized service for maintaining configuration information, naming, providing distributed synchronization, and providing group services. [1 mark]
\item Drill uses ZooKeeper to maintain cluster membership and health-check information. [1 mark]
\end{itemize}

\subsubsection*{b) Explain what is a Drillbit and how it functions. [3 marks]}

\begin{itemize}
\item A Drillbit is the process running on each active Drill node that coordinates, plans, and executes queries, as well as distributes query work across the cluster to maximize data locality. [1 mark]
\item The Drillbit receives the query from a client. A SQL parser in the DrillBit parses the SQL and form a logical plan. [0.5 mark]
\item The Drillbit sends the logical plan into a optimizer to optimize and convert the logical plan into a physical plan that describes how to execute the query. [0.5 mark]
\item A parallelizer in the Drillbit transforms the physical plan into multiple phases, called major and minor fragments.  [0.5 mark]
\item These fragments create a multi-level execution tree that rewrites the query and executes it in parallel against the configured data sources, sending the results back to the client or application. [0.5 mark]
\end{itemize}

\section*{Exercise 4 --- Simple Hadoop questions [5 marks]}

\subsection*{1. Why is ssh needed on the master and workers? How to configure it? [1 mark]}
\begin{itemize}
\item The master uses ssh protocol to send commands to the workers. [0.5 mark]
\item The worker should add the master's public key in ssh configurations. [0.5 mark]
\end{itemize}

\subsection*{Which Java version is needed by Hadoop, why? [1 mark]}
\begin{itemize}
\item Java version 8 is needed. [0.5 mark]
\item Some APIs are deprecated in the new versions of Java and Hadoop hasn't altered them yet. [0.5 mark]
\end{itemize}

\subsection*{Why should Hadoop's home be the same across the whole cluster? [1 mark]}
Because the master use the same configuration file to find the Hadoop's home on every worker.

\subsection*{How to use \texttt{hdfs dfs} command line interface to [2 marks]}

\subsubsection*{(i) list a directory [1 mark]}
\texttt{hdfs dfs -ls <hdfs path>}

\subsection*{(ii) upload or download a file [1 mark]}
\begin{itemize}
\item \texttt{hdfs dfs -put <local path> <hdfs path>} [0.5 mark]
\item \texttt{hdfs dfs -get <hdfs path> <local path>} [0.5 mark]
\end{itemize}


\end{document}

